\documentclass[11pt,a4paper,oneside]{article}
\usepackage[latin1]{inputenc}
\usepackage{amsmath}
\usepackage{amsfonts}
\usepackage{amssymb}
\usepackage{graphicx}
\usepackage{color}
\usepackage {tikz}
\usepackage{fancyvrb}
\usepackage{caption}
\usepackage{subcaption}
\usepackage{float}
\usetikzlibrary {er}
\usepackage[left=2.00cm, right=2.00cm, top=1.00cm]{geometry}
\graphicspath{{./}}
\fvset{tabsize=4}

\begin{document}
	\title{DS 200 - Research Methods \\ Module 3 - Research Seminars}
	\author{Shriram R. \\ M Tech (CDS) \\ 06-02-01-10-51-18-1-15763}
	\maketitle	
	
	\section{Accelerating Data Science at the Edge Using FPGAs - Dr. Viktor Prasanna - 04 Jan. 2019}
	
	The talk covered computational requirements of state-of-the-art problems in graph analytics, deep learning and autonomous energy grids and using FPGAs to accelerate these computations. Specifically, a problem on solar generation disaggregation which is about getting consumption and generation from solar irradiance, AMI data etc. was covered. Edge computing was discussed in relation to graph analytics, deep learning and different types of data. Several frameworks related to these areas were introduced. Different FPGA devices like FPGA++, Intel Stratix 10 and architectures were covered. Problems on partitioning and scheduling of heterogeneous devices was introduced and motivated. 
	
	
	\section{The Riscy Expedition - Prof. Arvind - 07 Jan. 2019}
	
	The talk discussed about leveraging open and free RISC-V ISA to support modular refinement, OOO (out-of-order) design and its impact on teaching. Speaker motivated about availability of large FPGAs in cloud and open source synthesis tools. Module composition using latency-insensitive (dataflow) framework and its limitations was discussed with an example. Hardware OO design using guarded atomic rules and cooperating Finite State Machines (FSM) was illustrated. The talk ended with the design of a complex superscalar processor design with non-blocking cache coherent memory system using conflict matrix and some case-studies from a related course.
	
	\section{Multi-Label Prediction - Dr. Yashaswi Verma  - 30 Jan. 2019}
	
	The talk was about predicting multi-labels in classification problems. The speaker highlighted the difference in multilabel problems where one or more labels are assigned or predicting the relevance of one or more labels is determined. The problem was motivated with several challenges related to incomplete labelling, missing ground truth, label correlations (directional/hierarchical), extreme scaling, feature and loss function selection. An order-free CNN-RNN for image annotation was introduced. It is a dictionary based vanilla RNN and fails to reflect true label dependencies. A new k-nearest neighbours based per-label algorithm was discussed.
	
	\pagebreak
	
	\begin{verbatim}
	
	
	
	\end{verbatim}
	
	\section{Simulating Human Vision and Vision Correcting Displays - Prof. Brian Barsky - 21 Mar. 2019}
	
	The talk focused on using vision correcting displays for rectifying eye anomalies unsolvable by optical. A primer on the structure of human eye was given. An accurate method for modelling human optical system using wave front aberrations was introduced. Different eye anomalies and their mathematical classification was discussed. A novel technique to rectify higher order aberrations by placing specially designed glass layer on top of existing displays was introduced. The layer would transform light from the display in a pre-determined manner to render a corrected input to eye and is customized to the type of aberration. 
	
	\section{Clustering Queries with Similar Intent at Web Scale - Dr. Manoj Agarwal - 05 Apr. 2019}
		
	\section{Unified analytics using Spark and CarbonData - Raghunandan and Venkata Ramana - 08 Apr. 2019}
	
	\section{The RAD-Squared Archetypal Pattern for Microservice Architectures - Dr. Zacharia George - 15 Apr. 2019}
	
	\section{Recovering Temporal Information in Dynamic Networks - Dr. Jithin K Sreedharan - 26 Apr. 2019}
	
    
\end{document}